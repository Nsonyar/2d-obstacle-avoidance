\section{Conclusion \label{Conclusion} }
This Bachelor thesis presented a self-supervised Machine Learning pipeline, used to develop an obstacle avoidance controller, which can maneuver an environment, with images as input alone.\\

The pipeline was implemented platform-independent, fully autonomous over all stages, with error recognition, consistency, and documentation functionalities. Each pipeline stage further provides excellent flexibility, maintainability, and possibilities to adjust or extend parameters by a centralized control mechanism easily.  The suggestion by \cite{Goodfellow-et-al-2016} to implement a functioning pipeline, very early in the training process, was beneficial to improve the project's performance early and achieve an efficient training of accurate models, iteratively.\\

The extensive analysis of the pre-training results provided a valuable and rewarding insight into the pipeline's performance, leading to applying the acquired knowledge for final training. The final results presented some efficient models, which underline the concept's effectiveness. The project shows that it is possible to gather data effectively, relate the acquired camera data with short-range information to train a model on the developed dataset, test the model and iteratively increase the model's performance. The final model can then be efficiently used to test and implement an obstacle avoidance controller, which can maneuver a simulation environment, by images from a camera alone.\\

The implementation also displayed disadvantages, which derive from the fact that just one camera is used as input. Future research might be rewarding to extend the current implementation with side cameras equipped to provide a more accurate controller, especially for the robot's outer ranges.
