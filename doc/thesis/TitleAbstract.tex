%Titelseite
\begin{titlepage}
\sffamily
\setlength{\tabcolsep}{0mm}
\begin{tabular*}{\textwidth}{l@{\extracolsep\fill}r} 

%\hspace{-0.4cm}
%\includegraphics[width=6cm]{Bilder/logo_welle_en}  % Englische Version des Logos 
\includegraphics[width=5cm]{Bilder/rwu} % Deutsche Version des Logos

  &
\raisebox{3mm}{
	\begin{tabular}{r}
%\rule{0cm}{0.5cm}
Course Applied Computer Science\\[0.5mm]
School of Electrical engineering and Computer Science\\
\end{tabular}}
\end{tabular*}
\setlength{\tabcolsep}{6pt}

\vspace*{4cm}
\begin{center}
\textbf{\Large{Bachelor-Thesis}}\\
\vspace*{1cm}
\textbf{\LARGE{Self-Supervised Machine Learning Pipeline for Obstacle Avoidance}}\\
\vspace*{2cm}
\large{for the purpose of obtaining the degree}\\[2mm]
\large{Bachelor of Computer Science}\\
\end{center}

%\vfill
\vspace{2cm}
\begin{center}

	submitted by:\\[5mm]
{\Large Martin Samuel Lanz} \\[5mm]
    \today \\[3cm]
{\normalsize
	\begin{tabular}{rl}
	1. Gutachter: 	& Prof. Dr. Markus Schneider\\
	2. Gutachter: 	& M.Sc. Daniel Hofer 
	\end{tabular}
}
\end{center}

\end{titlepage}


%Eidesstattliche Erkl�rung
\begin{newpage}
\vspace*{\fill}
\section*{Eidesstattliche Erkl�rung}
Hiermit versichere ich, die vorliegende Arbeit selbststndig und unter ausschlielicher Verwendung der angegebenen Literatur und Hilfsmittel erstellt zu haben. Die Arbeit wurde bisher, in gleicher oder hnlicher Form, keiner anderen Prfungsbehrde vorgelegt und auch nicht verffentlicht.\\\\

\vspace{3cm}
\begin{tabular*}{\textwidth}{c@{\extracolsep\fill}cc}
\cline{1-1}
\cline{3-3}
\\
\ \ \ \ \ \ \ \ \ Unterschrift\ \ \ \ \ \ \ \ \ \ & & \ \ \ \ \ \ \ \ \ Ort, Datum\ \ \ \ \ \ \ \ \ \\
\end{tabular*}
\end{newpage}



%Vorwort, Zusammenfassung(Abstract), Danksagung(Acknowledgements)
\begin{newpage}
\vspace*{\fill}
\section*{Abstract}
This Bachelor thesis proposes a self-supervised Machine Learning pipeline for obstacle avoidance, where a 2-dimensional image serves as the input, and a laser provides the relevant ground truth necessary for training.\\

Self-supervision has the advantage that no human labeling is necessary to gather data for training. The project will show that this is possible with a simple camera and a laser and refer to other projects with similar approaches.\\

The goal is further to implement as many pieces of a machine learning pipeline, as autonomous as possible, where a user can start a robot in a given environment, gather data, use that data for training, and then operate on the deployed model.\\

I have chosen this topic, not because of merely personal interests in machine learning, but also as an addition to our area of specialization and as an extension to previous classes like Autonomous Mobile Robots or Artificial Intelligence.\\

\section*{Acknowledgments}
Throughout the writing of this thesis, I have recevied a great deal of support and assistance.\\

I would first like to express my gratitude to Mr. Daniel Hofer M.Sc, and Mr. Benjamin St�hle M.Sc., who constantly gave me excellent support and advice along this project.\\

I would also like to thank my first supervisor, Professor Dr. Markus Schneider, for his suggestions and support.\\

I also want to thank my father, for his great support. Without him, it would have been much more challenging or even impossible, to study and organize everything which comes along, while I could focus on school.\\

Furthermore, I would like to thank my wife Sonia Rufino Castro de Lanz, for her generous assistance, not just while writing this thesis, but also for all the years of support throughout the studies.
\end{newpage}


\newpage
